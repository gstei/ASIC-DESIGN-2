\subsection{Known Limitations}

\subsubsection{SPI Register Addressing Off-by-One Error}

\subsubsection{Current Measurement Inaccurate if $V_{DDL} \neq V_{IN}$}

The internal $I_L$ inductor current measurement circuitry operates on the principle, that it amplifies the voltage drop over the input PMOS transistor caused by its $R_{DS,on}$ and using it as a shut resistor. The output of this approach however is only valid when the transistor is in conduction mode. In the phase where the transistor is non-conducting, the circuit would ordinarily still amplify the voltage differential between source and drain, which is significant, and leads to an output corresponding to a large inductor current. To combat this we implemented a circuit to short the amplifier inputs when the input transistor is non-conducting leading to an output corresponding to no inductor current. As the regulator implements peak-current mode control, the 0 current reading in the non-conducting phase leads to no problems and allows the circuit to start in the correct operating point when conduction starts. \\
In the implementation of the circuit we mistakenly shorted the inverting amplifier input $V_M$ with $V_{DDL}$ instead of $V_{IN}$ which is connected to the non-inverting amplifier input $V_P$. In the simulation where we always used $V_{DDL} = V_{IN}$ this posed no problems as it does in practice when these conditions are applied. In cases where $V_{DDL} = V_{IN}$ is not applicable the current measurement in the non-conducting phase generates a inaccurate current reading of large magnitude. This applies for static condition where $V_{DDL} \neq V_{IN}$, e.g. $V_{DDL} = \qty{5}{\volt}; V_{IN}= \qty{4.3}{\volt}$, as well as for dynamic conditions such as when $V_{DDL} = V_{IN}$ set, but a large current transient of $I_L$ causes a voltage drop on $V_{DDL}$.\\
This issue could have various knock-on effects like the discontinuous regulation we observed in \autoref{sec:loadRegulation} but we could not prove this conclusively. The described issue could also exacerbate the issues with the start-up behavior as the large input current during start-up lead to a voltage differential between $V_{DDL}$ and $V_{IN}$. Compounding the large voltage transients on $V_{DDL}$ during start-up get differently attenuated by the off-chip bypassing on each power-rail leading to unmeasurable dynamic errors in the current measurement.

\subsubsection{Default Register Settings Disables Current Limit}

A misconfiguration of the default state in the internal registers turn's off the current limit for the converter. This causes the converter to start operating without an effective current limit and leads to the converter to pulling so much current that the supply collapses to under the limit set by the \ac{POR}. This behavior is documented in \autoref{sec:startup}. The current limit can be enabled after start-up and works as intended, but as the settings are stored in non-persistent memory, they are lost after restart and the chip therefore cannot start up with the current limit enabled. 


\subsubsection{Internal Current Reference out of Specification}

\subsubsection{Internal Oscillator out of Specification}