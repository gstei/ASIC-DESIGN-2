\section{Listings}
%Abkürzungsverzeichniss, für schön ausrichten längste Abkürzung in eckige Klammer
\subsection*{List of Abbreviations}
\vspace{0.1cm}
\begin{acronym}[MIMO]
%\acro{Abkürzung}{Ausdruck ausgeschrieben}
\acro{ASIC}{Application Specific Integrated Circuit}
%\acro{BJT}{Bipolar Junction Transistor}
%\acro{CCM}{Continous Conduction Mode}
%\acro{CMOS}{Complementary Metal–Oxide–Semiconductor}
%\acro{DCM}{Discontinous Conduction Mode}
\acro{DUT}{Device Under Test}
%\acro{ESD}{Electrostatic Discharge}
%\acro{FET}{Field-Effect Transistor}
%\acro{FSM}{Finite-State Machine}
%\acro{GgNMOS}{Gate-Ground-NMOS}
%\acro{HBM}{Human Body Model}
\acro{HI}{Hearing Instruments}
\acro{IC}{Integrated Circuit}
%\acro{I2C}{Inter-Integrated Circuit}
\acro{LDO}{Low Drop-Out}
%\acro{MOS}{Metal-Oxide Semiconductor}
%\acro{MOSFET}{Metal-Oxide Semiconductor Field-Effect Transistor}
%\acro{MPW}{Multi Project Waver}
%\acro{NMOS}{N-Type Metal-Oxide Semiconductor}
%\acro{OTA}{Operational Transconductance Amplifier}
%\acro{PFM}{Pulse-Frequency Modulation}
%\acro{PMOS}{P-type Metal-Oxide Semiconductor}
\acro{POR}{Power-on-Reset}
%\acro{PWM}{Pulse-Width Modulation}
%\acro{SCR}{Silicon Controlled Rectifier}
%\acro{SEPIC}{Single-Ended Primary-Inductor Converter}
\acro{SPI}{Serial Peripheral Interface}
%\acro{TSBB}{Two Switch Buck-Boost}
%\acro{TSD}{Thermal Shutdown}
%\acro{TVS}{Transient-Voltage-Suppression}
%\acro{UART}{Universal Asynchronous Receiver/Transmitter}
\acro{USB}{Universal Serial Bus}
\end{acronym}	
\clearpage
\pagebreak


\nomenclature[A, 02]{\(c\)}{\href{https://physics.nist.gov/cgi-bin/cuu/Value?c}
	{Speed of light in a vacuum}
	\nomunit{\SI{299792458}{\meter\per\second}}}
\nomenclature[A, 03]{\(h\)}{\href{https://physics.nist.gov/cgi-bin/cuu/Value?h}
	{Planck constant}
	\nomunit{\SI[group-digits=false]{6.62607015e-34}{\joule\per\hertz}}}
\nomenclature[A, 01]{\(G\)}{\href{https://physics.nist.gov/cgi-bin/cuu/Value?bg}
	{Gravitational constant} 
	\nomunit{\SI[group-digits=false]{6.67430e-11}{\meter\cubed\per\kilogram\per\second\squared}}}
\nomenclature[A, 01]{\(m_e\)}{\href{https://physics.nist.gov/cgi-bin/cuu/Value?me}
	{Electron mass} 
	\nomunit{\SI[group-digits=false]{9.1093837015e-31}{\kilogram}}}
\nomenclature[A, 01]{\(\epsilon_0\)}{\href{https://physics.nist.gov/cgi-bin/cuu/Value?ep0}
	{Vacuum electric permittivity} 
	\nomunit{\SI[group-digits=false]{8.8541878128e-12}{\farad\per\meter}}}
\nomenclature[A, 01]{\(\mu_0\)}{\href{https://physics.nist.gov/cgi-bin/cuu/Value?mu0}
	{Vacuum magnetic permeability} 
	\nomunit{\SI[group-digits=false]{1.25663706212e-6}{\newton\per\ampere\squared}}}
\nomenclature[A, 01]{\(e\)}{\href{https://physics.nist.gov/cgi-bin/cuu/Value?e}
	{Elementary charge} 
	\nomunit{\SI[group-digits=false]{1.602176634e-19}{\coulomb}}}
\nomenclature[B, 03]{\(\mathbb{R}\)}{Real numbers}
\nomenclature[B, 02]{\(\mathbb{C}\)}{Complex numbers}
\nomenclature[B, 01]{\(\mathbb{H}\)}{Quaternions}
\nomenclature[C]{\(V\)}{Constant volume}
\nomenclature[C]{\(\rho\)}{Friction index}

\printnomenclature
\clearpage
\pagebreak

%Abbildungsverzeichnis
\subsection*{\listfigurename}
\renewcommand*\listfigurename{}
\listoffigures\thispagestyle{fancy}
\clearpage
\pagebreak

%Tabellenverzeichnis
\subsection*{\listtablename}
\renewcommand*\listtablename{}
\listoftables\thispagestyle{fancy}
\clearpage
\pagebreak

%Quellenverzeichnis
\subsection*{Bibliography}


%\bibliography{Literatur}
%\bibliographystyle{plain}
\printbibliography[heading=none]

%=[heading=subbibliography, keyword=lit, title={Quellenverzeichnis}]
%\printbibliography[heading=subbibliography, keyword=manual, title={Datenblätter}]
%\printbibliography[heading=subbibliography, keyword=online, title={Onlinequellen}]
%\printbibliography[heading=subbibliography, keyword=abb, title={Bildquellen}]

\clearpage