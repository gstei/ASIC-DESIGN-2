\section{Conclusion}
\label{chap:conclustion}

%We were able to achieve our goal of designing a highly integrated buck-boost converter \ac{IC}, which meets the specifications given to us by the Sonova AG. The peak-current control based converter is able to accurately create a output voltage both higher and lower than its input. The output voltage can be maintained over the whole specified input voltage range and regulator can quickly react to changes on the input and output. To ensure save operation the design features a foldback over-current protection, which leads to a short circuit safe output and basic soft-start functionality. To protect the switching transistor from over-temperature induced failure, we implemented thermal shutdown mechanism, which safely shuts off the converter if the transistors get to hot. We were not able to meet the deadline for the the tape-out, as we had to change the converter topology late in the process and due to the significant time required in the layout stage of the design.\\\\
%We faced a steep learning curve as both of us were unfamiliar we the tools as we started the project. Consequently, our initial timeline was to optimistic and had to be adjusted as the design and layout efforts required more time than anticipated. We sincerely appreciate the opportunity given to us by the university and department IMES to work on such an \ac{ASIC} project. However, we acknowledge the scope of the project was overly ambitious for a master's project, resulting in a significant higher time investment than anticipated. Therefore, we recommend to our supervisor and expert that future projects should be smaller in scale, particularly if the participants are not proficient users of the design tools required, as a substantial amount of time was dedicated to learning the program and troubleshooting errors, rather than focusing on design aspects.\\\\
%Having completed the pre-layout phase, we have successfully defined all the parameters necessary for measuring the ASIC. The system design simulation has indicated that the \ac{ASIC} should function as intended, giving us confidence that the final ASIC will work accordingly and we can proceed as planned.

%The chip we created mostly works as intended we are very proud of this achievement. As can be seen in 
During this second phase of this project we successfully finished the chip design and were able to meet the tape-out deadline. The weeks leading up to this deadline were hectic and some of the mistakes described in \autoref{sec:limitations} could probably have been avoided with more time. In preparation to receiving the samples we created the test setup \autoref{sec:testsetup} creating both custom hardware as well as software for semi-automated chip testing from scratch allowing us to quickly start testing as soon as the samples arrived. After some initial problems with the bring-up mainly caused by problems stemming from \autoref{subsubsec:cur_mes_inac} we are able to perform some automated test we prepared as well as more in depth manual testing of select samples with an Oscilloscope. The automated tests lead to the device characterization described in \autoref{sec:characteristics} and the manual measurements were used for diagnosing the issues described in \autoref{sec:limitations} and generate insights detailed in \autoref{sec:results}. \\

We are overall very proud of what we were able to achieve, given that the device mostly works as expected and this being the first chip we ever taped-out. While the mistakes in the current measurement circuit and in the default register configurations are disappointing, the high level functionality of the chip is preserved and we able to complete a large set of the tests we planed to conduct. Our test setup worked as designed with minimal need for fixes to the hardware or software after initial testing with the commercial chip. \\

The samples we characterized largely conform with the simulated values. We were however not able to find the reason behind the out of specification current reference and oscillator blocks. For both we lack adequate access to internal signals showing a lack of foresight with respect to \ac{DFT}. This once again stems from the tight timeline leading up to the tape-out, leaving us with minimal time to carefully consider with signal to lead off-chip and to implement the required \ac{DFT} structures for accurate probing. We are pleased to see the buck-boost converter working as intended given our concerns with the current handling capabilities of the power stage and given the overall complexity of the regulator consisting exclusively custom designed IP blocks.






\clearpage