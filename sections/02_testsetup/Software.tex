

\subsection{Software}

\subsubsection{Architecture}
The architecture of the test software was designed with a modular approach in mind, incorporating separate classes for each test and measurement instrument. This design choice not only simplifies the extension of the software with new instruments and tests but also enables looping over different tests without the constant need to initialize and close measurement instruments.

For the data storage a database solution was selected providing for a convenient method for storing information and easily allowing for subsequent data analysis. SQLite was selected as the database of choice due to its lightweight nature and user-friendly interface, eliminating the need for a running server.

The database was structured with the following columns:
\begin{itemize}
    \item \textbf{Id}: The unique identifier for each measurement
    \item \textbf{chip\_id}: The identifier for the chip that was measured
    \item \textbf{measurement\_type}: The type of measurement performed
    \item \textbf{parameter1}: The first parameter used for the test, such as the input voltage applied
    \item \textbf{parameter2}: The second parameter used for the test, such as the load applied to the output
    \item \textbf{temperature}: The temperature of the chip during the measurement
    \item \textbf{data}: The measured data
    \item \textbf{measurement\_result}: The result of the measurement, if applicable
    \item \textbf{Timestamp}: The timestamp of the measurement
\end{itemize}

\subsubsection{Test Software Language}
The test software was implemented using Python. Python is a high-level programming language widely utilized in the scientific community and increasingly in testing due to its ease of learning and the availability of drivers for nearly every measurement instrument \cite{Wikipedia:Python}. Additionally, one of the authors had prior experience in writing test scripts in Python for chip verification and validation, enabling the initiation of test writing without a significant investment in learning a new programming language and environment.

\subsubsection{Test Software Implementation}
Prior to commencing the test software development, the various tests to be performed were determined, aiming to align with those conducted in simulations. These included:
\begin{itemize}
    \item Startup behavior with reset disabled
    \item Startup behavior with active reset and subsequently disabled reset
    \item Load step response with load variations of \qty{1}{\milli\ampere}, \qty{100}{\milli\ampere}, and \qty{200}{\milli\ampere}
    \item Output response to stepped input voltage changes
    \item Turn-off and turn-on behavior with brief reset enablement
    \item SPI test to mux out the wanted analog/digital signal, change the clock frequency, and read back the register values
\end{itemize}

\subsubsection{Test Setup}
The primary test setup utilized the PXI system from National Instruments (NI). This system integrates various measurement instruments into a single unit and serves as the computer running the Python code. Consequently, there are minimal delays due to additional wires between the computer and measurement instruments. Moreover, synchronized triggers are available on this measurement system, eliminating the need for trigger wiring between instruments. This advantage reduces setup complexity and wiring requirements. The specific PXI system employed was the PXI model \glqq{}NI PXIe-8881\grqq{}, equipped with the following modules:

\begin{itemize}
    \item \textbf{PXIe-4141}: This is a SMU (Source Measurement Unit) which can be used to apply the input voltage to the chip
    \item \textbf{E3631A}: This is a power supply which can be used to apply the input voltage to the chip
    \item \textbf{PXI-5142}: This is a two channel oscilloscope used to measure the output and input voltage of the chip
    \item \textbf{PXI-5163}: This is a two channel oscilloscope used to measure the output and input current of the chip
    \item \textbf{TP04300}: This is the thermostreamer used to control the temperature of the chip
    \item \textbf{PXI-6363}: This is a GPIO controller used to control the reset of the chip and the different load resistors
    \item \textbf{NGE-100}: This is a power supply to provide the power for the test circuit
    \item \textbf{FTDI C232HM-DDHSL-0}: This is a USB to SPI converter used to communicate with the chip
\end{itemize}

An overview of all the instruments can be seen in Figure \autoref{fig:overview}. More details about the hardware can also be found in the \href{https://github.com/gstei/asic_validation/}{git repository}\footnote{https://github.com/gstei/asic\_validation.git} where test was written.

\begin{figure}[h]
    \centering
    \includegraphics[width=0.8\textwidth]{../ASIC-DESIGN-2/images/02_test_setup/01_overview.jpg}
    \caption{Overview of the test setup}
    \label{fig:overview}
\end{figure}

\subsubsection{Github}
The test software was written using Git as its version control system, with the repository being hosted on GitHub. This choice was driven by GitHub's widespread adoption and its seamless integration for collaborative code sharing among team members. Furthermore, Git's robust features allow for efficient tracking of changes and management of various software versions. The repository can be found at \href{https://github.com/gstei/asic_validation.git}{the following link}\footnote{https://github.com/gstei/asic\_validation.git}. Additionally the repository uses github actions to automatically run tests on every push to the repository. This ensures that the code is written in a appealing way according to the PEP8 standard and that the tests are most probably running without any errors, for that \href{par:Flake8}{flake8} and \href{par:Pylint}{pylint} were used as linter and integrated in the github workflow.

\paragraph{Pylint}
\label{par:Pylint}
Pylint is a static code analysis tool for Python, adhering to the style guidelines outlined in PEP 8. It checks various aspects of Python code, including line length, variable naming conventions, and interface implementation consistency. Pylint is similar to Pychecker and Pyflakes but offers additional features such as generating UML diagrams using the Pyreverse module. It can be used independently or integrated into various IDEs and editors like Eclipse with PyDev, Spyder, Visual Studio Code, Atom, GNU Emacs, and Vim\cite{Wikipedia:Pylint}.

\paragraph{Flake8}
\label{Par:Flake8}
Flake8 is a Python linting tool that scans Python codebases for errors, style inconsistencies, and complexity. It consists of three underlying tools: PyFlakes for error checking, McCabe for complexity analysis, and pycodestyle for style conformity with PEP8 guidelines. Flake8 stands out due to its extensive plugin ecosystem, allowing users to augment its capabilities and address a wide range of issues and concerns in Python code\cite{flake8}.


\subsubsection{SPI Interface}
\label{subsubsec:SPI}
The chip's registers are controlled via an SPI interface, necessitating an SPI Master. For the SPI master the FTDI C232HM-DDHSL-0 USB to SPI converter was chosen, primarily due to its user-friendly nature and the availability of Python drivers. 

A discrepancy between the SPI mode 1 implemented on the chip and the SPI mode 1 on the FTDI Chip was however found. On the FTDI Chip, the CS and the first SPI clock edge are activated simultaneously, which was not the case on the test bench used in the simulation to verify the digital part of the chip. Due to its implementation, the finite state machine only operates correctly when the CS is activated before the first clock edge, as illustrated in \autoref{fig:spi_modes} (CPOL=0, CPHA=1). 

To address this, we implemented a custom driver in software, which operates at 1kHz instead of 500kHz. Given that we only need to write/read seven different registers, this slower speed does not significantly impact the overall performance, and the difference is imperceptible to the human operator. 

However, for future projects, we recommend considering the test instruments during the chip design phase to avoid such issues. The SPI interface should function correctly with a standard microcontroller that uses the SPI standard, as shown in \autoref{fig:spi_modes}, where the CS is activated before the first clock edge. However, it's important to note that there are some unique implementations of the SPI, as in the FTDI C232HM-DDHSL-0, which should also be considered during the design phase.

\begin{figure}[h]
    \centering
    \includegraphics[width=0.8\textwidth]{../ASIC-DESIGN-2/images/04_spi/SPI_timing_diagram_CS.svg.png}
    \caption{SPI Modes \cite{Wikipedia:SPI}}
    \label{fig:spi_modes}
\end{figure}
\subsubsection{Conclusion}
In retrospective it was a very good decision to spend quite some time in the software architecture in the beginning of the project and to use a database. Since it turned out that once it is clear what and how something has to be implemented the software can be written quite fast and with the code checker in the background also with a reasonable quality, this for example allowed to add the thermostreamer to the tests within two hours without any issues and all measurement results were available in the database. Nevertheless it also turned out that when the chip does not behave as expected the test script is not useful at all. Due to that the automated tests only worked with the TI chip, but not with our chip since we had startup problems on our chip which caused that the whole automated test could not be executed since the chip started not correctly as also mentioned in 
\autoref{subsubsec:cur_mes_inac}.


