

\subsection{Software}

\subsubsection{Architecture}
For the architecture of the test software, it was decided to adopt a modular approach with separate classes for each test and measurement instrument. This architecture facilitates easy extension of the software with new instruments and tests. Additionally, it allows looping over different tests without the need to initialize and close measurement instruments every time they are used. For data storage, the decision was made to employ a database, as it offers a convenient way to store and analyze data afterward. Thereby SQLite was chosen as the database solution due to its lightweight nature and ease of use, requiring no server to be running. The database was structured with the following columns:
\begin{itemize}
    \item \textbf{Id}: The unique identifier for each measurement.
    \item \textbf{chip\_id}: The identifier for the chip that was measured.
    \item \textbf{measurement\_type}: The type of measurement performed.
    \item \textbf{parameter1}: The first parameter used for the test, such as the input voltage applied.
    \item \textbf{parameter2}: The second parameter used for the test, such as the load applied to the output.
    \item \textbf{temperature}: The temperature of the chip during the measurement.
    \item \textbf{data}: The measured data.
    \item \textbf{measurement\_result}: The result of the measurement, if applicable.
    \item \textbf{Timestamp}: The timestamp of the measurement.
\end{itemize}

\subsubsection{Test Software Language}
The test software was implemented using Python. Python is a high-level programming language widely utilized in the scientific community and increasingly in testing due to its ease of learning and the availability of drivers for nearly every measurement instrument\cite{Wikipedia:Python}. Additionally, one of the authors had prior experience in writing test scripts in Python for chip verification and validation, enabling the initiation of test writing without a significant investment in learning a new programming language and environment.

\subsubsection{Test Software Implementation}
Prior to commencing the test software development, the various tests to be performed were determined, aiming to align with those conducted in simulations. These included:
\begin{itemize}
    \item Startup behavior with reset disabled.
    \item Startup behavior with active and subsequently disabled reset.
    \item Load step response with load variations of 1mA, 100mA, and 200mA.
    \item Output response to stepped input voltage changes.
    \item Turn-off and turn-on behavior with brief reset enablement.
\end{itemize}

\subsubsection{Test Setup}
The primary test setup utilized the PXI system from National Instruments (NI). This system integrates various measurement instruments into a single unit and serves as the computer running the Python code. Consequently, there are no delays due to additional wires between the computer and measurement instruments. Moreover, synchronized triggers are available on this measurement system, eliminating the need for trigger wiring between instruments. This advantage reduces setup complexity and wiring requirements. The specific PXI system employed was the PXI model \glqq{}NI PXIe-8881\grqq{}, equipped with the following modules:

\begin{itemize}
    \item \textbf{PXIe-4141}: This is a SMU (Source Measurement Unit) which can be used to apply the input voltage to the chip.
    \item \textbf{E3631A}: This is a power supply which can be used to apply the input voltage to the chip.
    \item \textbf{PXI-5142}: This is a two channel oscilloscope used to measure the output and input voltage of the chip.
    \item \textbf{PXI-5163}: This is a two channel oscilloscope used to measure the output and input current of the chip.
    \item \textbf{TP04300}: This is the thermostreamer used to control the temperature of the chip.
    \item \textbf{PXI-6363}: This is a GPIO controller used to control the reset of the chip and the different load resistors.
    \item \textbf{NGE-100}: This is a power supply to provide the power for the test circuit.
\end{itemize}

An overview of all the instruments can be seen in Figure \autoref{fig:overview}.

\begin{figure}[h]
    \centering
    \includegraphics[width=0.8\textwidth]{../ASIC-DESIGN-2/images/02_test_setup/01_overview.jpg}
    \caption{Overview of the test setup}
    \label{fig:overview}
\end{figure}

\subsubsection{Github}
The test software was written using Git as its version control system, with the repository being hosted on GitHub. This choice was driven by GitHub's widespread adoption and its seamless integration for collaborative code sharing among team members. Furthermore, Git's robust features allow for efficient tracking of changes and management of various software versions. The repository can be found at \href{https://github.com/gstei/asic_validation.git}{the following link}\footnote{https://github.com/gstei/asic\_validation.git}. Furthermore the repository uses github actions to automatically run tests on every push to the repository. This ensures that the code is written in a nice way according to the PEP8 standard and that the tests are most probably running without any errors, for that \href{par:Flake8}{flake8} and \href{par:Pylint}{pylint} were used as linter and integrated in the github workflow.

\paragraph{Pylint}
\label{par:Pylint}
Pylint is a static code analysis tool for Python, adhering to the style guidelines outlined in PEP 8. It checks various aspects of Python code, including line length, variable naming conventions, and interface implementation consistency. Pylint is similar to Pychecker and Pyflakes but offers additional features such as generating UML diagrams using the Pyreverse module. It can be used independently or integrated into various IDEs and editors like Eclipse with PyDev, Spyder, Visual Studio Code, Atom, GNU Emacs, and Vim\cite{Wikipedia:Pylint}.

\paragraph{Flake8}
\label{Par:Flake8}
Flake8 is a Python linting tool that scans Python codebases for errors, style inconsistencies, and complexity. It consists of three underlying tools: PyFlakes for error checking, McCabe for complexity analysis, and pycodestyle for style conformity with PEP8 guidelines. Flake8 stands out due to its extensive plugin ecosystem, allowing users to augment its capabilities and address a wide range of issues and concerns in Python code\cite{flake8}.

