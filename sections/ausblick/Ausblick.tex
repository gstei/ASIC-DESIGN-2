\section{Outlook}
\label{chap:outlook}
As described, the designed \ac{ASIC} achieves its main objectives and largely performs up to specification. It is therefore suitable as a basis for further design iterations or as a reference for incorporating some of the IP created into other designs. The identified limitations would need to be taken into consideration and changes would need to be implemented to remedy them. While the root cause for the current reference offset could not be found in the time attributed for troubleshooting, the issue does not appear to be insurmountable or a blocker for a redesign. The issues described with the digital circuitry could probably also be remedied with a limited amount of effort in a redesign. We therefore could forsee a continuation of this design.\\
The test setup could be further refined to allow for more in-depth testing as some manual testing steps carried could reasonably be automated. This would provide more detailed insights into the performance of the chip. For instance, the effects of the switching frequency on efficiency and power loss could be further investigated or the effects of temperature on the regulation characteristics could be recorded. In any case the test setup provides an extensible framework for \ac{ASIC} validation and could be used for validation of other similar designs. \\
Finally, the project has demonstrated the potential of custom \ac{ASIC}s in the field of hearing instrument charging. With further development and refinement, these chips could play a crucial role in improving the efficiency and reliability of \ac{HI} charging cradles. By incorporating the opportunities provided by high levels of integration the size of circuits can significantly be decreased while maintaining an extensive feature set. 
\clearpage