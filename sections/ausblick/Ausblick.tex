\section{Outlook}
\label{chap:outlook}
\subsection{Time-plan}
Since the pre-layout phase is finished with this document. The next important phase is the layout phase and then the tape-out. According to our actual plan which can also be found online under the following \href{https://asic-project.atlassian.net/jira/software/projects/MAP/boards/3/timeline}{link}\footnote{https://asic-project.atlassian.net/jira/software/projects/MAP/boards/3/timeline} the tape out is planned in October 2023. So that we get the first physical \ac{ASIC}'s somewhere in April. During the time period where we wait on the \ac{ASIC}'s we will already prepare the test script and adapter PCBs to test all the specified parameters on the chip.
\subsection{Hours}
\href{subsec:jira}{Jira}, which allows time tracking, was utilized by one project participant to monitor the hours dedicated to specific tasks. This feature provides a high-level overview, as shown in  \autoref{fig:project_hours}, depicting the distribution of hours across different project segments, referred to as epics in \href{subsec:jira}{Jira}. It is important to note that this data reflects the input from only one participant, as the other participant ceased reporting his hours. Notably, a significant portion of time was invested in system design. However, it is worth mentioning that this category encompasses time spent on debugging and configuring Cadence for accurate simulations. Additionally, 108 hours have been allocated to layout work, although only a few blocks have been completed thus far. Considering future planning, it is evident that the period leading up to tapeout will be demanding, leaving no room for additional implementations on the ASIC before tape-out.
%\begin{figure}[htbp]
%	\centering
%	%trim=left botm right top
%	\includegraphics[trim=1 1 1 1,clip,width=\textwidth]{images/project_hours.png}
%	\caption{Project hours and distribution of one participant only so far (05.07.2023)}
%	\label{fig:project_hours}
%\end{figure}
\clearpage