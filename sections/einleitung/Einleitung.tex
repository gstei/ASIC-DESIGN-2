\section{Introduction}
\label{chap:introduction}
The goal of this project was to create prototype \ac{ASIC} for use inside a charging cradle of a hearing aid. It solves the problem of creating a stable and accurately regulated \qty{5}{\volt} from any USB specification compliant power source. This entails being able to boost the input voltage up from \qty{4.35}{\volt} as well as being able to buck it down from \qty{5.5}{\volt} and regulate to a stable \qty{5}{\volt} from any voltage in between. This requirement comes from the fact that modern \ac{HI}'s have a lithium battery inside, which have a charging cut-off voltage of \qty{4.2}{\volt}. In the case with a \qty{4.35}{\volt} supply voltage and the use of an \ac{LDO} regulator to charge with battery, the dropout voltage of the regulator can not be maintained leading to a loss of output regulation. Therefore the battery will not be able to be fully charged or could only be charged at a reduced rate. The resistive losses stemming from the contact resistances of the charger pins further increase the headroom required. A stable voltage is therefore required to charge the \ac{HI} at full speed, as the contact resistances can be in the order of several ohms, which causes a significant voltage drop. \cite{analog_devices_usb_charging} \\\\
The project of designing and testing such an \ac{ASIC} was divided in two parts, a \glqq pre-layolut phase\grqq{} and a \glqq post-layolut phase\grqq{}, since the project was executed in a master program which requires two separate stages. This report will focus on the \glqq post-laylout phase\grqq{} as the \glqq pre-layolut phase\grqq{} was already documented in the previous report. In the first chapter we represent parts of the tape-out process with an emphasis on the design changes since the last report and a high-level overview of the designed \ac{IC}. Thereafter we present the test setup created in order to validate the design. The test setup contains custom \ac{PCB}s for the characterization and custom software for automated testing. We used the test setup to characterize the samples we received and present the results in the subsequent chapter. Where applicable comparisons are made between the simulated results and the real world hardware. In the last chapter we go in depth into the limitations with the chip we discovered and provide analysis of the root cause.\\\\
%\textcolor{red}{TBD}
\clearpage