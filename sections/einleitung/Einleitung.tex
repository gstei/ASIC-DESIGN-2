\section{Introduction}
\label{chap:introduction}
%The goal of this project is to create prototype \ac{ASIC}, which can be used inside a charging cradle of a \ac{HI}. Thereby the \ac{ASIC} should deliver a constant voltage of 5V to reliably charge the \ac{HI} from any \ac{USB} supply. Which means according to the \ac{USB} specification the circuit should be fully functional down to a voltage of \qty{4.35}{\volt}. The reason of this requirement comes from the fact that the new \ac{HI}'s have a lithium battery inside, which require a charging voltage of up to \qty{4.2}{\volt}. When the \ac{USB} supply is then as low as \qty{4.35}{\volt} the battery will not fully charge any more or only very slow, since the \ac{LDO} regulator has a certain headroom and the charger also contains some contact resistances from the charger pins to the \ac{HI} pins. So if one wants to charge at full speed and one assumes that the charging resistance can be several ohms this causes a significant voltage drop. \cite{analog_devices_usb_charging} \\\\
%The project of designing and testing such an \ac{ASIC} was divided in two parts \glqq pre-layolut phase\grqq{} and \glqq post-layolut phase\grqq{}, since the project was executed in a master program which requires two separate stages.
\textcolor{red}{TBD}
\clearpage