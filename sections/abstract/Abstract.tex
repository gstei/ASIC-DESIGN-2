\section*{\huge Abstract}
This document presents the results of a project aimed at designing an \ac{ASIC} for a charging cradle for \ac{HI} from Sonova AG. The main objective of the \ac{ASIC} was to safely charge the HI from a standard USB power supply. The \ac{ASIC} was designed to charge two HIs simultaneously at their maximum charging speed, and included a serial port for external configuration.

The project was divided into two phases: the pre-layout phase which was already documented in the previous year and the post-layout phase. The pre-layout phase involved creating the specifications and a system design with ideal components. These components were then replaced with implementable designs from libraries or custom-made designs, while still meeting the specifications. 
The post-layout phase involved creating a layout for the ASIC, manufacturing and packaging the chip, and validating and characterizing the samples.

The results of the project showed that the \ac{ASIC} mostly worked as intended, with some limitations. The chip was able to meet the tape-out deadline, and the test setup created for the project allowed for testing of the samples. The automated tests and manual measurements provided valuable insights into the performance of the chip. However, there were some issues with the current measurement circuit and the default register configurations. Despite these issues, the high-level functionality of the chip was preserved, and a large set of the planned tests were completed.



\label{chap:abstract}
%This document focuses on the post-layout phase of a project, which was divided into pre-layout and post-layout stages. The project was initiated by Sonova AG, with the objective of designing a compact 200 mA DC/DC converter for potential integration into the charging case of their hearing devices. This second part of the documentation provides a comprehensive overview of the key steps undertaken during the post-layout phase, including DC/DC topology selection, circuit architecture/design, and chip interface considerations. By examining these crucial aspects, this documentation offers valuable insights into the initial phase of the project, providing a solid foundation for the subsequent layout and manufacturing, and testing stages.
\clearpage